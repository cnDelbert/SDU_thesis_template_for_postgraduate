% !Mode:: "TeX:UTF-8"
% usersettings.tex
\usepackage{paralist}
\usepackage{dtklogos}
\usepackage{xcolor}
\usepackage{listings}
% Add Chapter to TOC && Remove Contents from TOC
\usepackage[chapter, nottoc]{tocbibind}
% 允许使用子图 subfloat
\usepackage{subfig}
% 允许使用图文混排
\usepackage{wrapfig}
% picinpar 和 picins 都无法成功实现图文混排。
% \usepackage{picinpar}
% \usepackage{picins}
% 使用 ccmap,则英文和数字都正常。如果论文重复率高的话,注释掉 ccmap ,英文和数字被映射为汉字,相当于增加干扰码,降低重复率。
\usepackage{ccmap}

% \newtheorem{环境名}[编号延续]{显示名}[编号层次]
% 在下例中,我们定义了四个环境:定义、定理、引理和推论,它们都在一个section内编号,而引理和推论会延续定理的编号。
% \newtheorem{definition}{定义}[section]
% \newtheorem{theorem}{定理}[section]
% \newtheorem{lemma}[theorem]{引理}
% \newtheorem{corollary}[theorem]{推论}

\lstset{language=TeX}
% \lstset{extendedchars=false}
\lstset{breaklines}	
\lstset{numbers=left,numberstyle=\tiny,%commentstyle=\color{red!50!green!50!blue!50},
frame=shadowbox,rulesepcolor=\color{red!20!green!20!blue!20},escapeinside=``,
xleftmargin=2em,xrightmargin=2em,aboveskip=1em,backgroundcolor=\color{red!3!green!3!blue!3},
basicstyle=\small\ttfamily,stringstyle=\color{purple},keywordstyle=\color{blue!80}\bfseries,
commentstyle=\color{olive}}
\numberwithin{footnote}{page}
\renewcommand{\thefootnote}{\arabic{footnote}}
\renewcommand{\CTeX}{\SHUANG{C}\TeX}
